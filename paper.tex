\documentclass[a4paper,12pt]{article}
\usepackage{amsmath,amssymb,amsfonts,graphicx,xcolor}
\usepackage{tikz,pgfplots}
\usepackage{geometry}
\usepackage{hyperref}
\ProvidesPackage{paper_theme}

% Page Layout
\RequirePackage{geometry}
\geometry{a4paper, margin=1in}

% Colors
\RequirePackage{xcolor}
\definecolor{background}{HTML}{1E1E1E} % Dark gray
\definecolor{textcolor}{HTML}{D4D4D4} % Light gray
\definecolor{accentcolor}{HTML}{569CD6} % Blue
\definecolor{mathcolor}{HTML}{CE9178} % Orange

% Apply Colors
\pagecolor{background}
\color{textcolor}

% Font
\RequirePackage{lmodern}
\renewcommand{\familydefault}{\sfdefault}

% Hyperlinks
\RequirePackage{hyperref}
\hypersetup{
	colorlinks=true,
	linkcolor=accentcolor,
	urlcolor=accentcolor,
	citecolor=accentcolor
}

% Math Formatting
\RequirePackage{amsmath, amssymb}
\everymath{\color{mathcolor}}

% Title Formatting
\RequirePackage{titlesec}
\titleformat{\section}{\color{accentcolor}\Large\bfseries}{\thesection}{1em}{}
\titleformat{\subsection}{\color{accentcolor}\large\bfseries}{\thesubsection}{1em}{}
\titleformat{\subsubsection}{\color{accentcolor}\normalsize\bfseries}{\thesubsubsection}{1em}{}

% Table of Contents
\RequirePackage{tocloft}
\renewcommand{\cftsecfont}{\color{accentcolor}\bfseries}
\renewcommand{\cftsubsecfont}{\color{accentcolor}}
\renewcommand{\cftsubsubsecfont}{\color{accentcolor}}

% Listings (for Code)
\RequirePackage{listings}
\lstset{
	backgroundcolor=\color{background},
	basicstyle=\ttfamily\color{textcolor},
	keywordstyle=\color{accentcolor},
	stringstyle=\color{mathcolor},
	commentstyle=\color{gray},
	numberstyle=\color{gray},
	breaklines=true
}

% Graphics
\RequirePackage{graphicx}
\RequirePackage{pgfplots}
\pgfplotsset{compat=1.18}
\usetikzlibrary{shadows,shapes,positioning,arrows.meta}

\title{\textbf{Comprehensive Guide to Calculating Derivatives}}
\author{Badger Code}
\date{\today}

\begin{document}
\maketitle

\section{Introduction}
Derivatives are fundamental in calculus, allowing us to understand rates of change. This paper explores different methods of computing derivatives: the limit definition, L'Hôpital's Rule, rule-based differentiation, and forward-mode automatic differentiation.

\section{Derivative Definition via Limits}
The derivative of a function $f(x)$ at a point $x=a$ is defined as:
\begin{equation}
	f'(a) = \lim_{h \to 0} \frac{f(a+h) - f(a)}{h}
\end{equation}
To illustrate, consider $f(x) = x^2$. Using the limit definition:
\begin{align*}
	f'(x) &= \lim_{h \to 0} \frac{(x+h)^2 - x^2}{h} \\
	&= \lim_{h \to 0} \frac{x^2 + 2xh + h^2 - x^2}{h} \\
	&= \lim_{h \to 0} (2x + h) = 2x.
\end{align*}
Thus, $\frac{d}{dx} x^2 = 2x$.

\section{L'Hôpital's Rule}
If a function results in an indeterminate form such as $\frac{0}{0}$ or $\frac{\infty}{\infty}$, L'Hôpital's Rule states:
\begin{equation}
	\lim_{x \to a} \frac{f(x)}{g(x)} = \lim_{x \to a} \frac{f'(x)}{g'(x)},
\end{equation}
provided the limit on the right exists.

Consider:
\begin{equation}
	\lim_{x \to 0} \frac{\sin x}{x}.
\end{equation}
Since both numerator and denominator approach zero:
\begin{equation}
	\lim_{x \to 0} \frac{\sin x}{x} = \lim_{x \to 0} \frac{\cos x}{1} = 1.
\end{equation}

\section{Rule-Based Differentiation}
Several fundamental differentiation rules simplify computations:
\begin{itemize}
	\item \textbf{Power Rule:} $\frac{d}{dx} x^n = nx^{n-1}$.
	\item \textbf{Product Rule:} $(fg)' = f'g + fg'$.
	\item \textbf{Quotient Rule:} $\left(\frac{f}{g}\right)' = \frac{f'g - fg'}{g^2}$.
	\item \textbf{Chain Rule:} $\frac{d}{dx} f(g(x)) = f'(g(x)) g'(x)$.
\end{itemize}

Example: Compute $\frac{d}{dx} (x^3 \sin x)$. Using the product rule:
\begin{align*}
	(x^3 \sin x)' &= (x^3)' \sin x + x^3 (\sin x)' \\
	&= 3x^2 \sin x + x^3 \cos x.
\end{align*}

\section{Forward-Mode Automatic Differentiation}
Automatic differentiation (AD) uses dual numbers to compute derivatives efficiently. A dual number is defined as:
\begin{equation}
	x_\epsilon = x + \epsilon dx, \quad \text{where } \epsilon^2 = 0.
\end{equation}
For a function $f(x)$, we substitute $x_\epsilon$:
\begin{equation}
	f(x_\epsilon) = f(x) + \epsilon f'(x).
\end{equation}
Thus, $f'(x)$ is extracted directly.

Example: Compute $f(x) = x^2$ using dual numbers:
\begin{align*}
	f(x_\epsilon) &= (x + \epsilon)^2 = x^2 + 2x \epsilon + \epsilon^2 \\
	&= x^2 + 2x \epsilon. \quad \text{(Since $\epsilon^2 = 0$)}
\end{align*}
Hence, $f'(x) = 2x$.

\section{Conclusion}
This paper covered multiple derivative computation techniques, from the fundamental limit definition to advanced automatic differentiation. Each method serves different purposes, from theoretical rigor to computational efficiency.

\end{document}